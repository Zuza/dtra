\documentclass[times, utf8, diplomski]{fer}
\usepackage{booktabs}
\usepackage[utf8]{inputenc}

\usepackage[T1]{fontenc}
\usepackage{fixltx2e}
\usepackage{graphicx}
\usepackage{longtable}
\usepackage{float}
\usepackage{wrapfig}
\usepackage{soul}
\usepackage{textcomp}
\usepackage{marvosym}
\usepackage{wasysym}
\usepackage{latexsym}
\usepackage{amssymb}
\usepackage{hyperref}
\usepackage{caption}


\begin{document}

\title{LISA}

\author{Filip Paveti\'{c}, Goran \v{Z}u\v{z}i\'{c}}

\maketitle

% Dodavanje zahvale ili prazne stranice. Ako ne želite dodati zahvalu, naredbu ostavite radi prazne stranice.
\zahvala{}

\tableofcontents

\chapter{Uvod}
Ovaj rad opisuje sustav za mapiranje kratkih DNA sekvenci na referentni genom. Rad je podijeljen na 4 dijela:
\begin{itemize}
\item u prvom dijelu ćemo napraviti kratki uvod u probleme bioinformatike i pozadinu problema kojeg sustav rješava
\item drugi dio se bavi modelom, algoritmom i implementacijom nad kojim je sustav ostvaren
\item slijede rezultati testiranja i usporedba s najpoznatijim analognim alatima
\item naposlijetku dajemo kratki zaključak i smjernice za daljnji rad
\end{itemize}

Sustav je razvijen u sklopu natjecanja identifikacije organizama iz toka DNA sekvenci koje je objavila tvrtka \emph{InnoCentive} pod 
sponzorstvom vlade Sjedinjenih Američkih država i koje je trenutno u tijeku.

\section{Biološka pozadina problema}
bla bla bla

\chapter{Sustav}

\chapter{Zaključak}
Zaključak.

\bibliography{literatura}
\bibliographystyle{fer}

\begin{sazetak}
Sažetak na hrvatskom jeziku.

\kljucnerijeci{Ključne riječi, odvojene zarezima.}
\end{sazetak}

% TODO: Navedite naslov na engleskom jeziku.
\engtitle{Title}
\begin{abstract}
Abstract.

\keywords{Keywords.}
\end{abstract}

\end{document}
