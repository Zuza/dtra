\documentclass[times, utf8, diplomski]{fer}
\usepackage{booktabs}
\usepackage[utf8]{inputenc}

\usepackage[T1]{fontenc}
\usepackage{fixltx2e}
\usepackage{graphicx}
\usepackage{longtable}
\usepackage{float}
\usepackage{wrapfig}
\usepackage{soul}
\usepackage{textcomp}
\usepackage{marvosym}
\usepackage{wasysym}
\usepackage{latexsym}
\usepackage{amssymb}
\usepackage{hyperref}
\usepackage{caption}
\usepackage[usenames, dvipsnames]{color}

\begin{document}

\title{LISA}

\author{Filip Paveti\'{c}, Goran \v{Z}u\v{z}i\'{c}}

\maketitle

% Dodavanje zahvale ili prazne stranice. Ako ne želite dodati zahvalu, naredbu ostavite radi prazne stranice.
\zahvala{}

\tableofcontents

\chapter{Uvod}
Ovaj rad opisuje sustav za mapiranje kratkih DNA sekvenci na referentni genom. Rad je podijeljen na 4 dijela:
\begin{itemize}
\item u prvom dijelu ćemo napraviti kratki uvod u probleme bioinformatike i pozadinu problema kojeg sustav rješava
\item drugi dio se bavi modelom, algoritmom i implementacijom nad kojim je sustav ostvaren
\item slijede rezultati testiranja i usporedba s najpoznatijim analognim alatima
\item naposlijetku dajemo kratki zaključak i smjernice za daljnji rad
\end{itemize}

Sustav je razvijen u sklopu natjecanja identifikacija organizama iz toka DNA sekvenci
kojeg je objavila tvrtka \emph{InnoCentive} pod sponzorstvom vlade Sjedinjenih Američkih Država
i koje još nije završilo \footnote{više informacija o samom natjecaju možete pronaći na \emph{http://tinyurl.com/dna-dtra}}.

\section{Bioinformatika}

Bioinformatika je u posljednjih nekoliko godina u centru ogromne pažnje i entuzijazma do kojih su
doveli napretci u razumijevanju živih bića i brzom razvoju tehnologije pomoću kojih ih možemo proučavati.
Dosada se bioinformatika koristila u svrhe proučavanja nasljednih svojstava, evolucije između vrsta i identifikacije
patogenih bakterija, dok se u bliskoj budućnosti vjeruje kako će pomoći u proizvodnji boljih lijekova koji bi bili
posebno prilagođeni zaraženoj osobi.

Molekularna biologija i biokemija su nam otkrili kako najveću funkcionalnu ulogu u gotovo svim biološkim procesima
živih organizama imaju molekule zvane \emph{proteini}, linearni nizovi aminokiselina spojenih peptidnim vezama.
Njihova uloga u organizmima seže od katalizacije procesa (npr. kod metabolizma), signalizacije i adhezije stanica
sve do imunološkog sustava. Proteini su konstruirani zasebno u svakoj stanici od svojih gradivnih elemenata (aminokiselina),
a kod koji specificira način na koji se grade proteini se nalazi u masivnoj staničnoj molekuli
deoksiribonukleinske kiseline (DNA).

Dominantno područje bioinformatike se bavi sekvenciranjem (čitanjem) i analizom DNA. Molekula DNA je dugi linearni niz sastavljen od otprilike
3 milijarde\footnote{kod ljudi} povezanih parova baza nukleinskih kiselina. Veličina tih brojeva stvara potrebu za razvojem računalnih alata
bez kojih bilo kakva obrada postaje nemoguća.

Nedavni razvoj moderne tehnologije je doveo do strmoglavog pada cijene sekvenciranja molekule DNA do te razine da je
postalo jasno kako će u bliskoj budućnosti najsporiji i najteži dio bioinformatike biti u konstrukciji efikasnih
algoritama koji brzo i pouzdano obrađuju sve veču i veću količinu informacija na njihovom raspolaganju.

\section{DNA i mutacije}
Kao što je spomenuto, DNA modeliramo kao linearni niz povezanih parova baza. U DNA se pojavljuju točno četiri različite baze i njih
označavamo slovima \emph{A, G, T} i \emph{C}\footnote{skraćeno od adenin, guanin, timin i citozin, vrsta nukleinskih kiselina koje se pojavljuju}.
Na taj način DNA modeliramo dugim nizom slova nad gornjom četvoroslovnom abecedom, primjerice:

\begin{verbatim}
... AGTGAGGAAAAAAAAAGGTCAATGCAGCACTTGAGCCAACATTGTAGAT
    GTTGTACTGCAAGGTCAGGTCTCGCCCCTCCACGGCGTATCTGTTCAG
    CAGTGACTTGGAGGCAAGAAAATCAAACCCGTGATCGATGGTACCGAGC ...
\end{verbatim}

Kroz vrijeme se na molekuli DNA javljaju razne mutacije. Biolozi su izolirali nekoliko dominantnih mutacija:

\begin{enumerate}
\item Supstitucija - zamjena jedne baze drugom. Najčešća od navedenih mutacija.
  Uzrokovana greškama u replikaciji i kemikalijama.
  \nopagebreak
  \begin{figure}[!ht]
    \begin{center}
      A C G \textcolor{blue}{T} T G A C \\
      A C G \textcolor{red}{A} T G A C
      \caption{Primjer supstitucije}
    \end{center}
  \end{figure}

\item Ispuštanje i umetanje - iz DNA se ukloni ili doda slijed baza. Ova mutacija je vrlo štetna i obično
  rezultira gubitkom funkcionalnosti tog dijela gena.

  \nopagebreak
  \begin{figure}[!ht]
    \begin{center}
      A C G \textcolor{red}{T T G} A C \\
      A C G A C
      \caption{Primjer ispuštanja}
    \end{center}
  \end{figure}
\item Udvostručavanje - ponavljanje dijela sekvence dva ili više puta zaredom. Ova mutacija je iznimno rijetka,
  ali se smatra da je unatoč tomu imala veliku ulogu u povećanju genetskog koda živih bića.
  \begin{figure}[!ht]
    \begin{center}
      A C G \textcolor{blue}{T T G A} C \\
      A C G \underline{\textcolor{red}{T T G A}} \underline{\textcolor{red}{T T G A}} C \\
      \caption{Primjer udvostručavanja}
    \end{center}
  \end{figure}
  
\item Inverzija - okretanje dijela sekvence. Iznimno rijetka mutacija koja se najčešće zanemaruje pri analizi.
  \begin{figure}[!ht]
    \begin{center}
      A C \textcolor{blue}{T C A A} G G \\
      A C \textcolor{red} {A A C T} G G
      \caption{Primjer inverzije}
    \end{center}
\end{figure}
\end{enumerate}



\section{Sekvenciranje DNA}

\chapter{Sustav}

\chapter{Zaključak}
Zaključak.

\bibliography{literatura}
\bibliographystyle{fer}

\begin{sazetak}
Sažetak na hrvatskom jeziku.

\kljucnerijeci{Ključne riječi, odvojene zarezima.}
\end{sazetak}

% TODO: Navedite naslov na engleskom jeziku.
\engtitle{Title}
\begin{abstract}
Abstract.

\keywords{Keywords.}
\end{abstract}

\end{document}
