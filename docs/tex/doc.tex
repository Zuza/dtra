\documentclass[times, utf8, diplomski]{fer}
\usepackage{booktabs}
\usepackage[utf8]{inputenc}

\usepackage[T1]{fontenc}
\usepackage{fixltx2e}
\usepackage{graphicx}
\usepackage{longtable}
\usepackage{float}
\usepackage{wrapfig}
\usepackage{soul}
\usepackage{textcomp}
\usepackage{marvosym}
\usepackage{wasysym}
\usepackage{latexsym}
\usepackage{amssymb}
\usepackage{hyperref}
\usepackage{caption}


\begin{document}

\title{LISA}

\author{Filip Paveti\'{c}, Goran \v{Z}u\v{z}i\'{c}}

\maketitle

% Dodavanje zahvale ili prazne stranice. Ako ne želite dodati zahvalu, naredbu ostavite radi prazne stranice.
\zahvala{}

\tableofcontents

\chapter{Uvod}
Ovaj rad opisuje sustav za mapiranje kratkih DNA sekvenci na referentni genom. Rad je podijeljen na 4 dijela:
\begin{itemize}
\item u prvom dijelu ćemo napraviti kratki uvod u probleme bioinformatike i pozadinu problema kojeg sustav rješava
\item drugi dio se bavi modelom, algoritmom i implementacijom nad kojim je sustav ostvaren
\item slijede rezultati testiranja i usporedba s najpoznatijim analognim alatima
\item naposlijetku dajemo kratki zaključak i smjernice za daljnji rad
\end{itemize}

Sustav je razvijen u sklopu natjecanja identifikacije organizama iz toka DNA sekvenci koje je objavila tvrtka \emph{InnoCentive} pod 
sponzorstvom vlade Sjedinjenih Američkih država i koje je trenutno u tijeku.

\section{Biološka pozadina problema}
bla bla bla

\chapter{Pristupa}

\section{Pregled}
Ovo poglavlje počinjemo kratkih pregledom trendova u razvoju mappera te kasnije dajemo detaljan opis našeg pristupa. Jedan od prvih algoritama za poravnanje očitanja je algoritam globalnog poravnanja pod nazivom Smith-Waterman (TODO referenca). Rješenje dobiveno tim algoritmom može se smatrati veoma točnim zbog toga što ima statistički opravdano značenje, međutim zbog velike vremenske složenosti algoritma (TODO napisati složenosti) nije praktično za korištenje.\\
Većina boljih modernih mappera (poput SNAP-a (TODO referenca) i SeqAlto-a (TODO referenca)) koriste tkzv. seed-and-extend pristup. U tom pristupu prvo se od referentnog genoma gradi indeks koji mapira niz od svakih uzastopnih k-znakova genoma na sve pozicije unutar genoma na kojima se taj podniz nalazi. Ukupan broj takvih različitih podnizova je $4^k$, što je i prva praktična opaska, jer za $k\le32$ možemo takav podniz zakodirati u $64$-bitni cjelobrojni tip podataka. Očitanja se obrađuju na način da se ponovno promatraju podnizi uzastopnih znakova duljine $k$ te se vrši upit u prethodno izgrađeni indeks. Pozicije dobivene takvim upitom daju kandidatne pozicije u čijoj se okolini nalazi potencijalno poravnanje očitanja na genom. Razlike između mappera ovog tipa su primarno u načinu izgradnje indeksa i načinu ocjene kvalitete poravnanja na nekoj kandidatnoj poziciji. Tako primjerica SNAP koristi hash tablicu kao strukturu podataka nad kojom je izgrađen indeks, dok SeqAlto koristi sortirano polje. Za ocjenu kvalitete SNAP koristi \emph{edit-distance} metriku, dok SeqAlto koristi općenitiju \emph{Needleman-Wunsch} metriku. Oba pristupa daju značajno poboljšanje efikasnosti u odnosu na pristup spomenut na početku ove sekcije, međutim ti su mapperi i dalje nepraktični budući da kroz vrijeme strojevi za izradu očitanja napreduju i duljine očitanja postaju sve veće čime pretpostavke koje osiguravaju efikasnost ovih alata postaju prejake.\\
Naš mapper indeks izgrađuje po uzoru na SeqAlto, međutim uvodimo novu metriku ocjene kvalitete koja za red složenosti poboljšava postojeće algoritme bez gubitka točnosti što ćemo potkrijepiti eksperimentima u kasnijim poglavljima.
Metrika kvalitete bazira se na traženju najduljeg uzlaznog podniza\footnote{engl. Longest Increasing Subsequence; odatle dolazi inspiracija za naziv LISA - LIS Aligner}\\
\\
TODO: neka slika koja reprezentira DNA+indeks

\section{sad}

\chapter{Implementacija}

\chapter{Rezultati}

\chapter{Zaključak i daljnji rad}
Zaključak.

\chapter{Dodatak}

\bibliography{literatura}
\bibliographystyle{fer}

\begin{sazetak}
Sažetak na hrvatskom jeziku.

\kljucnerijeci{Ključne riječi, odvojene zarezima.}
\end{sazetak}

% TODO: Navedite naslov na engleskom jeziku.
\engtitle{Title}
\begin{abstract}
Abstract.

\keywords{Keywords.}
\end{abstract}

\end{document}
