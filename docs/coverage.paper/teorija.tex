% !TEX TS-program = pdflatex
% !TEX encoding = UTF-8 Unicode

% This is a simple template for a LaTeX document using the "article" class.
% See "book", "report", "letter" for other types of document.

\documentclass[11pt]{article} % use larger type; default would be 10pt

%\usepackage[utf8]{inputenc} % set input encoding (not needed with XeLaTeX)
\usepackage[T1]{fontenc}
\usepackage{fixltx2e}
\usepackage{graphicx}
\usepackage{longtable}
\usepackage{float}
\usepackage{wrapfig}
\usepackage{soul}
\usepackage{textcomp}
\usepackage{marvosym}
\usepackage{wasysym}
\usepackage{latexsym}
\usepackage{amssymb}
\usepackage{hyperref}
\usepackage{caption}
\usepackage{pdfpages}
\usepackage{float}
%\usepackage[usenames, dvipsnames]{color}
\usepackage{listings}
\usepackage{xcolor}
\usepackage{subfig}
\usepackage{algpseudocode}
\usepackage[Algoritam]{algorithm}
\usepackage{amssymb}

%%% Examples of Article customizations
% These packages are optional, depending whether you want the features they provide.
% See the LaTeX Companion or other references for full information.

%%% PAGE DIMENSIONS
\usepackage{geometry} % to change the page dimensions
\geometry{a4paper} % or letterpaper (US) or a5paper or....
% \geometry{margin=2in} % for example, change the margins to 2 inches all round
% \geometry{landscape} % set up the page for landscape
%   read geometry.pdf for detailed page layout information

\usepackage{graphicx} % support the \includegraphics command and options

% \usepackage[parfill]{parskip} % Activate to begin paragraphs with an empty line rather than an indent

%%% PACKAGES
\usepackage{booktabs} % for much better looking tables
\usepackage{array} % for better arrays (eg matrices) in maths
\usepackage{paralist} % very flexible & customisable lists (eg. enumerate/itemize, etc.)
\usepackage{verbatim} % adds environment for commenting out blocks of text & for better verbatim
\usepackage{subfig} % make it possible to include more than one captioned figure/table in a single float
% These packages are all incorporated in the memoir class to one degree or another...

%%% HEADERS & FOOTERS
\usepackage{fancyhdr} % This should be set AFTER setting up the page geometry
\pagestyle{fancy} % options: empty , plain , fancy
\renewcommand{\headrulewidth}{0pt} % customise the layout...
\lhead{}\chead{}\rhead{}
\lfoot{}\cfoot{\thepage}\rfoot{}

%%% SECTION TITLE APPEARANCE
\usepackage{sectsty}
\allsectionsfont{\sffamily\mdseries\upshape} % (See the fntguide.pdf for font help)
% (This matches ConTeXt defaults)

%%% ToC (table of contents) APPEARANCE
\usepackage[nottoc,notlof,notlot]{tocbibind} % Put the bibliography in the ToC
\usepackage[titles,subfigure]{tocloft} % Alter the style of the Table of Contents
\renewcommand{\cftsecfont}{\rmfamily\mdseries\upshape}
\renewcommand{\cftsecpagefont}{\rmfamily\mdseries\upshape} % No bold!

%%% END Article customizations

%%% The "real" document content comes below...

\title{Teorija}
\author{Filip Paveti\'{c}}
%\date{} % Activate to display a given date or no date (if empty),
         % otherwise the current date is printed 

\begin{document}
\maketitle

\section{LCS kao nositelj informacija za alignment}

\v{Z}elim pokazati da za referentni genom i read koji ima fiksan iznos gre\v{s}ke LCS  daje dovoljno informacije za pronalazak optimalnog smje\v{s}tanja reada na referentni genom obzirom na Needleman-Wunsch score (sa proizvoljnom matricom sli\v{c}nosti koju zovemo $similarity$\footnote{Ne bas bilo kakvom, jednom od onih spomenutih u literaturi. Literatura je trenutno kod mene na mailu.}). Pokazuje se da je dovoljno \v{c}uvati odreden broj najboljih poravnanja obzirom na LCS score kako bi taj popis sadr\v{z}avao najbolje smje\v{s}tanje obzirom na NW.\\
\\
Neka je poznato da je tijekom ocitavanja reada vjerojatnost pogreske $p_{err}$. Za sada pogresku ogranicujemo na supstitucije, ali napominjem da se slican racun moze napraviti i ako podrzimo brisanja/umetanja. Trenutno nisam znao kojim vrijednostima da ih scoream pa sam uzeo samo supstitucije. Duljinu reada oznacujemo s $N$. Tablica 1 daje vjerojatnosti pojavljivanja pojedine baze u ljudskom genomu.\\

\begin{table}[H]
\centering
\begin{tabular}{|c||c|c|c|c|c|c|}
\hline
	Baza & Vjerojatnost\\
\hline
\hline
	A & 0.30\\
\hline
	C & 0.20\\
\hline
	G & 0.20\\
\hline
	T & 0.30\\
\hline
\end{tabular}
\caption{Vjerojatnosti pojavljivanja pojedine baze - $p_{baza}$}
\end{table}

Buduci da pretpostavljamo poznat $p_{err}$, promatrajmo dvije vrste NW scorea. Prvi cemo zvati $NW_{idealni}$. U njemu supstitucije nece biti kaznjene (cijena supstitucije je 0). Drugi ce biti normalan NW score. Pogledajmo prosjecnu vrijednost za $NW_{idealni}$:

\begin{equation}
	NW_{idealni} = N * (1-p_{err}) * \sum\limits_{b=A,C,G,T} p_{baza}(b) * similarity(b,b)
\end{equation}

Gornjom jednadzbom odredjujemo da ce se tocno odredjena frakcija baza tocno poklapati i pridjeljujemo im prosjecan score za preklapanje. Definirajmo sada $NW_{normalni}$:

\begin{equation}
	NW_{normalni} = NW_{idealni} + N * p_{err} * prosjecna\ supstitucijska\ greska
\end{equation}

gdje je

\begin{equation}
	prosjecna\ supstitucijska\ greska = \frac{1}{c}\sum\limits_{i,j\in{A,C,T,G}, i\neq j} p_{baza}(i)*p_{baza}(j)*similarity(i,j)
\end{equation}

$c$ je normalizacijska konstanta za vjerojatnosti i iznosi:
\begin{equation}
	c = \sum\limits_{i,j\in{A,C,T,G}, i\neq j} p_{baza}(i)*p_{baza}(j)
\end{equation}

Primijetimo da je $NW_{normalni} \le NW_{idealni}$ (jer ce prosjecna supstitucijska greska biti negativna, vidi kako izgledaju matrice slicnosti u sljedecoj sekciji). Sada ce stvari postati malo trikaste i mozda sljedecih par recenica treba procitati 2-3 puta :).\\
\\
$NW_{idealni}$ je zapravo jednak LCS-u dvije sekvence pomnozenim sa prosjecnim scoreom tocnih matcheva (brojeva na dijagonali $similarity$-a. To je gornja granica na $NW_{normalni}$. Ono sto zelimo je naci i donju granicu, ali je zelimo moci izraziti preko LCS-a dvije sekvence, jer LISA racuna LCS. Znači trazimo donju granicu koja ima slican oblik kao $NW{idealni}$. Izrazimo to ovako:
$NW_{donja} = N *minimalni\ udio\ LCSa * \sum\limits_{b=A,C,G,T} p_{baza}(b) * similarity(b,b) $. Nadalje, mora vrijediti $NW_{donja} \le NW_{normalni}$. Buduci da je $minimalni\ udio\ LCSa$ nepoznanica koju trazimo, mozemo je izracunati tako da isforsiramo jednakost $NW_{donja}= NW_{normalni}$. Iz toga ispada:

\begin{equation}
	minimalni\ udio\ LCSa = \frac{NW_{normalni}}{NW_{idealni}}
\end{equation}

Sada smo dosli konacno do dijela kojeg smo htjeli pokazati, a to je koliko $NW_{donja}$ odstupa od $NW{idealni}$. To ce nam reci njihova relativna greska:

$relativna = \frac{NW_{idealni}-NW_{donja}}{NW_{idealni}}$

Primijetimo da relativnu gresku mozemo izracunati preko LCS-ova jer su ove velicine u takvom obliku.

\subsection{Supstitucijske matrice}

Necu ih tu prepisivati, ali imam clanak  koji sadrzi nekoliko razlicitih primjeraka na mailu. Isto tako, detalji se pronaci u kodu u $src/scripts/score\_diff.py$.

\subsection{Ra\v{c}unanje}

\begin{table}[H]
\centering
\begin{tabular}{|c||c|c|c|c|c|c|}
\hline
	sups. matrica / $p_{err}$ & 0.02 & 0.05 & 0.10 & 0.20\\
\hline
\hline
	CFTR & 0.02 & 0.05 & 0.10 & 0.23 \\
\hline
	HOXD & 0.02 & 0.05 & 0.11 & 0.24\\
\hline
	hum16pter & 0.02 & 0.05 & 0.10 & 0.24\\
\hline
\end{tabular}
\caption{$relativna$ za razlicite vrijednosti $p_{err}$ i razlicite supstitucijske matrice}
\end{table}
\end{document}
