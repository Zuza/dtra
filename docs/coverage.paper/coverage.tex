% !TEX TS-program = pdflatex
% !TEX encoding = UTF-8 Unicode

% This is a simple template for a LaTeX document using the "article" class.
% See "book", "report", "letter" for other types of document.

\documentclass[11pt]{article} % use larger type; default would be 10pt

\usepackage[utf8]{inputenc} % set input encoding (not needed with XeLaTeX)

%%% Examples of Article customizations
% These packages are optional, depending whether you want the features they provide.
% See the LaTeX Companion or other references for full information.

%%% PAGE DIMENSIONS
\usepackage{geometry} % to change the page dimensions
\geometry{a4paper} % or letterpaper (US) or a5paper or....
% \geometry{margin=2in} % for example, change the margins to 2 inches all round
% \geometry{landscape} % set up the page for landscape
%   read geometry.pdf for detailed page layout information

\usepackage{graphicx} % support the \includegraphics command and options

% \usepackage[parfill]{parskip} % Activate to begin paragraphs with an empty line rather than an indent

%%% PACKAGES
\usepackage{booktabs} % for much better looking tables
\usepackage{array} % for better arrays (eg matrices) in maths
\usepackage{paralist} % very flexible & customisable lists (eg. enumerate/itemize, etc.)
\usepackage{verbatim} % adds environment for commenting out blocks of text & for better verbatim
\usepackage{subfig} % make it possible to include more than one captioned figure/table in a single float
% These packages are all incorporated in the memoir class to one degree or another...

%%% HEADERS & FOOTERS
\usepackage{fancyhdr} % This should be set AFTER setting up the page geometry
\pagestyle{fancy} % options: empty , plain , fancy
\renewcommand{\headrulewidth}{0pt} % customise the layout...
\lhead{}\chead{}\rhead{}
\lfoot{}\cfoot{\thepage}\rfoot{}

%%% SECTION TITLE APPEARANCE
\usepackage{sectsty}
\allsectionsfont{\sffamily\mdseries\upshape} % (See the fntguide.pdf for font help)
% (This matches ConTeXt defaults)

%%% ToC (table of contents) APPEARANCE
\usepackage[nottoc,notlof,notlot]{tocbibind} % Put the bibliography in the ToC
\usepackage[titles,subfigure]{tocloft} % Alter the style of the Table of Contents
\renewcommand{\cftsecfont}{\rmfamily\mdseries\upshape}
\renewcommand{\cftsecpagefont}{\rmfamily\mdseries\upshape} % No bold!

%%% END Article customizations

%%% The "real" document content comes below...

\title{The Rising Intervals/Icy Tower Sweep}
\author{Filip Paveti\'{c}, Goran \v{Z}u\v{z}i\'{c}, Mile \v{S}iki\'{c}}
%\date{} % Activate to display a given date or no date (if empty),
         % otherwise the current date is printed 

\begin{document}
\maketitle

\section{Introduction}

DNA sequencing has got very big attention in the research community recently. The main reason is the advance in the DNA reading machines which made huge ammounts of data publicly available for scientists to analyze. With SNAP [TODO: referenca], SeqAlto[TODO: referenca] it seems that, to this date, problem of aligning short reads to a genome has been practically solved. Many of the existing tools fail processing long reads, mainly because calculating alignment using Needleman Wunsch [TODO: referenca] or Levenstein distance [TODO:referenca] is computationaly demanding.\\
\\
In this paper we analyze LCS\footnote{Longest common subsequence} as a metric for aligning long reads. We show that it is good metric for separating similar pair of DNA sequences from random pairs. Since LCS is not feasible for computing due to big complexity, we design a modification of the metric, called k-LCS, and we demonstrate its good performance on long read lengths. Aside from theoretical analysis, we apply the metric on real data and confirm its practical value.\\
\\
Before this work, LCS has been analysed by [TODO: reference]. TODO: napisati malo sto se sve zna o LCS-u. Mummer 1,2,3 [TODO: referenca] shares a similar intuition, although their use case is sligthly different and they do not provide theoretic justification of their approach (TODO: provjeriti da nemaju teorije, iako koliko se sjecam iz clanaka kazu da im je to uvjerenje).

\section{LCS}

\subsection{Analysis}

\section{k-LCS}

\subsection{Analysis}

\subsection{Computing k-LCS}

\section{Experiments}

\section{Future work}

There is ongoing work on an aligner using k-LCS as a basis. Experimental results demonstrated here were obtained using that aligner. TODO: ispricati kako je bazran na seed-and-extendu, ali s uzasno jednostavnim i neefikasnim indeksom.

\end{document}
